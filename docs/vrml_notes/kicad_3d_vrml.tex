\documentclass[a4paper, dvipdfm]{article}
\usepackage{graphicx}
\usepackage{longtable}

\setlength\oddsidemargin{-.25 in}
\setlength\textwidth{6.75 in}
\setlength\topmargin{-.5 in}
\setlength\textheight{9.75 in}

\makeatletter
\def\maketitle{%
  \null
  \thispagestyle{empty}%
  \vfill
  \begin{center}\leavevmode
    \normalfont
{\LARGE \@title\par}%
    \vskip 1cm
{\Large \@author\par}%
    \vskip 1cm
{\Large \@date\par}%
  \end{center}%
  \vfill
  \null
  \cleardoublepage
}
\makeatother

\pagestyle{myheadings}
\markboth{KiCAD 3D VRML Models, Python bindings}{}


\title{TUTORIAL\\
VRML2.0 and KiCAD}
\date{Version 0.2\\
15 Jan 2012}
\author{Dr. Cirilo Bernardo}

% Check the following in dip_pin.cpp:
%  p33z
%  p34y
%  p46y

\setlength{\topmargin}{0in}
\setlength{\headheight}{-0.5in}
%\setlength{\headsep}{0in}
\setlength{\textheight}{25cm}
\setlength{\textwidth}{16.5cm}
\setlength{\oddsidemargin}{0cm}
%\setlength{\evensidemargin}{2cm}
%\setlength{\parindent}{0.25in}
%\setlength{\parskip}{0.25in}

\begin{document}

\maketitle

% Revision history, if desired
%\pagenumbering{roman}
%\textbf{\Large{Revision History}}
%\vspace{1 ex}
%
%\begin{tabular}{|l|l|p{4.5in}|}
%\hline
%\textbf{Revision} & \textbf{Date} & \textbf{Comments}\\
%\hline
%RevA.1 & 2012-11-12 & Initial draft\\
%\hline
%\end{tabular}
%
%\clearpage

% start main numbering sequence
\setcounter{page}{1}
\pagenumbering{arabic}

\section{Introduction}
KiCad uses VRML97\footnote{\raisebox{-2ex}{\parbox{\textwidth}{http://www.web3d.org/x3d/specifications/vrml\\http://gun.teipir.gr/VRML-amgem/spec/vrmlspec.pdf}}}
to describe the 3D models used.
I would recommend reading the specification to gain a better understanding of it, but that is not essential to creating 3D models for KiCad.  In my
version of KiCad (0.0.20071129a-1, from the Debian Lenny archives) many features of the VRML specification are not supported and the 3D renderer also
has its peculiar non-conformant behavior.  This tutorial aims to provide enough information to be able to create VRML97 models which will be
correctly rendered by KiCad. Although Wings3D is often mentioned as a tool for creating the KiCad models, I find that Wings3D is not an appropriate
tool for the job.  KiCad makes use of a very small subset of the VRML specification and hand crafting the VRML document is not difficult.
In the rest of this section I make various comments about the VRML specification in relation to KiCad; most of it will be of no interest to
you unless you have read the standard but you must pay attention to the list of reserved keywords. The reserved keywords are always interpreted by the
VRML viewer in a specific way and must therefore only be used in accordance with the VRML specification.

VRML97 reserves the following keywords:

\begin{itemize}
\item {DEF}
\item {USE}
\item {IS}
\item {TO}
\item {PROTO}
\item {EXTERNPROTO}
\item {TRUE}
\item {FALSE}
\item {eventIn}
\item {eventOut}
\item {ROUTE}
\item {field}
\item {exposedField}
\end{itemize}

Of those keywords, I only use the following in creating 3D models for KiCad:
\begin{itemize}
\item {DEF}
\item {USE}
\end{itemize}

I have tested the PROTO feature and found that it does not work in KiCad; I have not tested the use of the other keywords in KiCad.

The following is a list of nodes, fields, and types which I have found to work with KiCad:

\begin{itemize}
\item Transform
  \begin{itemize}
    \item children
  \end{itemize}
\item Shape
  \begin{itemize}
    \item appearance
      \begin{itemize}
        \item material
      \end{itemize}
    \item geometry
      \begin{itemize}
        \item IndexedFaceSet
      \end{itemize}
  \end{itemize}
\end{itemize}

Within the geometry node I have tried the types Extrusion and IndexedFaceSet and only found IndexedFaceSet to work so this tutorial will
make exclusive use of the IndexedFaceSet to describe a model component.

\section{VRML Basics}
 A VRML file is a plain text file which uses utf8 text encoding; this encoding is similar enough to ASCII that we should rarely care about the differences.
We can create VRML files using plain text editors and view the files using free VRML viewers. I have used WhiteDune on Linux and Cortona3D on
MS Windows; I have also tried FreeWRL (version 1.22.10) but it had severe defects which made it unusable for verifying the KiCad models.
For editors, if you are developing on Windows, the Notepad editor works fine.

Every VRML97 file must begin with the line \verb~#VRML V2.0 utf8~ and the filename must end with `.wrl'. An object within the VRML file must
be associated with an appearance or else it will not be rendered. For KiCad the object's surfaces must be described using the IndexedFaceSet.
The following is a valid VRML file which we shall use as our starting point:

\begin{verbatim}
#VRML V2.0 utf8

# surface appearance of an IC pin
DEF mat_pin Appearance {
  material  Material {
    diffuseColor 0.73 0.73 0.75
    emissiveColor 0 0 0
    specularColor 0.73 0.73 0.75
    ambientIntensity 0.9
    transparency 0
    shininess 0.80
}}

Shape {
  appearance USE mat_pin
  geometry IndexedFaceSet {
    coord Coordinate { point [
      -1 -1 0, 1 -1 0, 1 1 0, -1 1 0
    ]}
    coordIndex [
      0, 1, 2, 3, -1
    ]
  }
}
\end{verbatim}

The \textbf{coord} field lists the points used to create a surface; the points are listed in triplets representing the X (left-right),
Y (down-up), and Z (back-front) dimensions of the display; the XYZ values are separated by a space and each coordinate point is
separated by a comma. The \textbf{coordIndex} field lists the elements of the \textbf{coord} field to produce \textbf{facets} (surfaces).
In this example we used elements 0, 1, 2, and 3 to create a single facet; each facet is composed of a list of element indices
which are separated by commas and the end of a set is represented by the index -1. In the example I use the \textbf{DEF} keyword
to define a material appearance which I named \textbf{mat\_pin}. Definitions allow us to reuse entities without typing the entire
description over and over again; this is a very useful feature to have when you are describing a 44-pin DIL package or perhaps a
416-pin BGA.  The material field uses an RGB color scheme to describe the color parameters of the surface; refer to the VRML
specification if you wish to learn more about the material parameters.

If you view the example you should see a large silver square with a surface texture which somewhat resembles that of a tinned IC pin.
If you rotate the square you will notice that it is not visible from the back side; if you imagine the points composing a facet, that facet
is visible from the direction in which the points form a counterclockwise sequence. In the example, change the index sequence to 3, 2, 1, 0, -1
and the square will now be visible from the other side only. If you want the square to be visible from both sides you can define two facets:

\begin{verbatim}
    coordIndex [
      0, 1, 2, 3, -1,
      3, 2, 1, 0, -1
    ]
\end{verbatim}

For the next exercise we will create a facet shaped like an hourglass; this exercise will illustrate a limitation of the rendering
routines and the means by which we can make the object look the way we intended. In the example below, the vertices of the hourglass
are enumerated in a fashion similar to the way we would read a page of text, beginning with the top left vertex and proceeding to the
right then moving down to the next vertex and going right. In this first part we will attempt to use all six points to create a single
facet.

\begin{verbatim}
#VRML V2.0 utf8

DEF mat_pin Appearance {
  material  Material {
    diffuseColor 0.73 0.73 0.75
    emissiveColor 0 0 0
    specularColor 0.73 0.73 0.75
    ambientIntensity 0.9
    transparency 0
    shininess 0.80
}}

Shape {
  appearance USE mat_pin
  geometry IndexedFaceSet {
    coord Coordinate { point [
      -2 2 0, 2 2 0, # (points 0,1) top part of hourglass
      -1 0 0, 1 0 0, # (points 2,3) middle of hourglass
      -2 -2 0, 2 -2 0, # (points 4,5) bottom of hourglass
    ]}
    coordIndex [
      0, 2, 4, 5, 3, 1, -1 # all points in a single facet
    ]
  }
}
\end{verbatim}

Now go and view that file; you will have to rotate the object to see what was rendered.  So what went wrong?
The facet which we attempted to create had concave features; for a facet to render correctly it must be convex.
Imagine drawing a line between every vertex in a facet. If any of those lines lie outside the facet then we
have a concave facet and it will not render correctly. In this instance it is easy to see that the line between
points 0 and 4 (the top and bottom left hand vertices) lies outside the hourglass. Concave facets must be
divided into convex facets to render our image correctly. In this example we simply divide the hourglass into
two trapezoidal facets using the following \textbf{coordIndex}:

\begin{verbatim}
    coordIndex [
      0, 2, 3, 1, -1, # top trapezoid
      2, 4, 5, 3, -1  # bottom trapezoid
    ]
\end{verbatim}

For the next example let's create a tetrahedron; type up the VRML file as it appears and supply your own facet
indices to get some practice in getting the facet to render as desired; remember that a facet is only visible
when the enumerated vertices are seen from a counterclockwise perspective. It was not necessary for me to
define an entity called \textbf{tetrahedron} (\verb~DEF tetrahedron~) in this example, but I put in the definition
so that we can play with the defined object later.

\begin{verbatim}
#VRML V2.0 utf8

DEF mat_pin Appearance {
  material  Material {
    diffuseColor 0.73 0.73 0.75
    emissiveColor 0 0 0
    specularColor 0.73 0.73 0.75
    ambientIntensity 0.9
    transparency 0
    shininess 0.80
}}

DEF tetrahedron Shape {
  appearance USE mat_pin
  geometry IndexedFaceSet {
    coord Coordinate { point [
      0 1.5 0,      # (point 0) top of tetrahedron
      -1 0 0,       # (point 1) bottom left
      0.5 0 0.866,  # (point 2) bottom right front
      0.5 0 -0.866, # (point 3) bottom right rear
    ]}
    coordIndex [
      (insert your facet sets)
    ]
}
}
\end{verbatim}

The following is just one of a number of valid facet sets which would render the tetrahedron as desired:

\begin{verbatim}
    coordIndex [
      0, 1, 2, -1, # front face
      0, 2, 3, -1, # right hand face
      0, 3, 1, -1, # rear face
      1, 3, 2, -1  # bottom face
    ]
\end{verbatim}

So far we have seen examples of \textbf{DEF} and \textbf{USE}, \textbf{Appearance}, and \textbf{Shape};
the last element I would like to introduce is the \textbf{Transform}. A Transform has the fields
\textbf{translation}, \textbf{rotation}, \textbf{scale}, and \textbf{children}.  A translation shifts
the coordinate origin relative to the current coordinate origin; transforms can be nested and each
nested transform may shift the origin relative to that of the previous transform.  The scale determines
the proportions of the X, Y, Z values relative to the current proportions and is applied \textbf{after}
the translation assigned to any particular transform; keep in mind that a scale will affect the
proportions of any translation within a nested transform. As with the translation, scales within nested
transforms are also cumulative. Rotations
define an axis of rotation using four parameters (X, Y, Z, angle). The angle is in radians and specifies
a counterclockwise rotation about the specified axis. For example, \verb~rotation 0 1 0 1.5707963~ will
rotate the coordinate system $\pi/2$ radians counterclockwise as seen when staring down the axis towards
the origin; the exact same rotation can be achieved with \verb~rotation 0 -1 0 -1.5707963~. 
A rotation is applied before translating the origin. Be careful when nesting transforms which have
a scale or rotation speficied.  When you use a transform you may specify at least one of translation,
rotation, or scale --- if none of these are specified then no transform is performed but the Transform
section is still valid VRML.

Add the following to the end of the tetrahedron file to see how \textbf{scale} is applied after \textbf{translation}
and how it also affects nested transforms.

\begin{verbatim}
Transform {
  translation 2 0 0
  children [
    USE tetrahedron
    Transform {
      scale 0.5 0.5 0.5
      translation 2 0 0
      children [
        USE tetrahedron
        Transform {
          scale 0.5 0.5 0.5
          translation 2 0 0
          children [
            USE tetrahedron
] } ] } ] }
\end{verbatim}

The following code can be added as well; it illustrates how \textbf{rotation} is applied before \textbf{translation}
and how it affects nested transforms.

\begin{verbatim}
Transform {
  translation 0 0 4
  rotation 0 1 0 0.5236
  children [
    USE tetrahedron
    Transform {
      translation 0 0 1
      rotation 0 1 0 0.5236
      scale 0.8 0.8 0.8
      children [
        USE tetrahedron
        Transform {
          translation 0 0 1
          rotation 0 1 0 0.5236
          scale 0.8 0.8 0.8
          children [
            USE tetrahedron
            Transform {
              translation 0 0 1
              rotation 0 1 0 0.5236
              scale 0.8 0.8 0.8
              children [
                USE tetrahedron
                Transform {
                  translation 0 0 1
                  rotation 0 1 0 0.5236
                  scale 0.8 0.8 0.8
                  children USE tetrahedron
} ] } ] } ] } ] }
\end{verbatim}

Play around a little and enjoy your basic VRML skills; soon we will jump into the details of creating a
3D model which KiCad can render. The tetrahedra in this short tutorial will not render in KiCad due to
limitations of its VRML engine, but what is important is that you develop some familiarity with writing
VRML files and are comfortable using the elements described so far. Unfortunately KiCad does not seem to
support \textbf{translation}, \textbf{rotation}, and \textbf{scale} as per the VRML specification so these
useful tools will not be available to us in describing a component.

\section{Elements of a VRML Description for KiCad}
In the previous section we created some VRML objects but never discussed the units used. According to the VRML97
specification one unit is one meter (39.370079 inches or 3.2808399 feet); that specification allows disparate
groups to create 3D objects with a uniform unit of size so that, for example, if one group created a realistic
house and another group created a realistic human then a third group can create a virtual world and the human
would fit nicely into the house without any fiddling about to determine appropriate scaling factors. KiCad
detracts from the VRML specification and uses a scale of 1 unit = 0.1 inches (2.54mm). This choice is rather
unfortunate since it makes the description of metric sized components a nuisance and we would need to use a
shrink ray on our board model before we can put the model into the hands of our virtual electronics hobbyist,
but those are some of the quirks of the implementation of KiCad.

To give our 3D models some aesthetic appeal we need to compromise between ease of description and rendering
complexity.  Increased complexity doesn't really matter if we're talking about a handful of components, but if you
were designing something fairly complex like a computer motherboard it's much nicer to work with, say, half a million
facets to render rather than 8 million facets. Components with very few facets simply do not look good. For example, you
can describe the case of a DIL package using a simple box (6 facets) but it will look nothing like a real
IC package because it does not have the notch (or if you wish, the dimple) and annoying rendering artifacts
will be obvious at the edges. If we bevel the edges of the box we get a much nicer representation with 26 facets
and although this is a huge improvement on the 6-facet box this still does not look so pleasing to the
board designer since there is still no notch. In the parametric description of a DIL package in a later section
we will use quite a few facets --- about 77 for the case and 48 for each pin, or 365 facets for a DIL6 and
2189 facets for a DIL44. Hopefully you will agree that we end up with a more beautiful description of the DIL packages
by employing so many bevels.

Before we design a component we must be familiar with how KiCad expects the VRML object to be oriented. For any
standards compliant VRML viewer, the X axis runs from left (negative) to right (positive), Y runs from the
bottom (negative) to the top (positive) and Z runs from the rear (negative) to the front (positive). When a
component is composed for KiCad, the default view in VRML will be a top view of the component rather than a
side view. If you designed a part so that the default view was a side view then you will need to perform a
remapping of coordinate points so that $X_1=X_0$, $Y_1=Z_0$, $Z_1=-Y_0$. Open up one of the wrl files supplied with KiCad so you
can see the orientation expected by KiCad. Using the file \verb~dil_8.wrl~ as an example, the top and
bottom of the case would be a positive Z value (meaning both are above the surface of the PCB) and the point at
which Pin 1 attaches to the case will meet the conditions $X<0$, $Y<0$, $Z>0$. KiCad expects the component to be centered
on $X=Y=0$ and, at least in the case of a DIL IC, Pin 1 to be at the bottom left corner of the default VRML view.

\subsection{Structure}
KiCad expects a VRML model to have a very specific structure; not all valid VRML documents will render
correctly in KiCad, but all KiCad 3D models are valid VRML.  In other words, we must follow a number of
restrictions when developing VRML models for KiCad. The general structure is as follows:

\begin{verbatim}
#VRML V2.0 utf8

DEF subpart_1 Transform {
  children [
    Shape {
      appearance Appearance {
        material DEF material_type Material {
          (material parameters)
        }
      }
      geometry IndexedFaceSet {
        coord Coordinate { point [
          (vertex list) ]
        }
        coordIndex [ (vertex indices per facet) ]
      }
    }
  ]
}

DEF subpart_2 Transform {
  children [
    Shape {
      appearance Appearance {
        material USE material_type
      }
      geometry IndexedFaceSet {
        coord Coordinate { point [
          (vertex list) ]
        }
        coordIndex [ (vertex indices per facet) ]
      }
    }
  ]
}
\end{verbatim}

\subsection{Transform, Children and Shape}
In the VRML standard the \textbf{Transform} node has the elements \textbf{translation},
\textbf{rotation}, \textbf{scale}, and \textbf{children}. In KiCad the transform must only
have a \textbf{children} list since the other elements are ignored.
The children list in turn may have shape nodes and other
transform nodes; however, to keep things simple I only use a single shape node within
the children list. A \textbf{Shape} node contains an \textbf{appearance} element and a
\textbf{geometry} element.

\subsection{Appearance and Materials}
The \textbf{appearance} element defines the material appearance of a surface. Although the VRML
standard permits a variety of options, for KiCad we will only use the \textbf{Material} node.
The \textbf{material} element and \textbf{Material} node describe the surface properties of
the facet including \textbf{diffuseColor} which is the reflective property of the surface when it is
not viewed at the same angle as the reflected light source, \textbf{emissiveColor} which is the
intensity of light emitted by an object (the value is usually set to 0 0 0 unless you want
a glow-in-the-dark component), \textbf{specularColor} which is the reflective property of the
surface when viewed at the angle of reflection of a light, \textbf{ambientIntensity} which is the
amount of ambient light blended in, \textbf{transparency}, and \textbf{shininess} which
controls the glossy appearance of the surface. VRML only supports the RGB color model and the
individual channel values must be in the range 0 to 1. Below are examples of materials which I have used
to describe a DIL case and its pins.

\begin{verbatim}
  material  DEF mat_pin Material {
    diffuseColor 0.73 0.73 0.75
    emissiveColor 0 0 0
    specularColor 0.73 0.73 0.75
    ambientIntensity 0.9
    transparency 0
    shininess 0.80
  }

  material DEF mat_case Material {
    diffuseColor 0.35 0.3 0.23
    emissiveColor 0 0 0
    specularColor 0.25 0.22 0.2
    ambientIntensity 0.5
    transparency 0
    shininess 0.05
  }
\end{verbatim}

\subsection{Geometry}
The \textbf{geometry} element in KiCad may only be of type \textbf{IndexedFaceSet} which must contain a
\textbf{coord} list of vertices and a \textbf{coordIndex} list of vertices composing each facet.
The \textbf{coord} element is a list of vertex coordinates in the order X Y Z and separated by spaces;
each vertex is separated from the next via a comma. The \textbf{coordIndex} element is a list of points
describing each facet; the index starts with 0, indices are separated by commas, and each set of facets
ends with the index -1. For example:

\begin{verbatim}
  geometry IndexedFaceSet {
    coord Coordinate { point [
        -1 -1 0, 1 -1 0, 1 1 0, -1 1 0,
        -1 -1 1, 1 -1 1, 1 1 1, -1 1 1 ]
    }
    coordIndex [
      3, 2, 1, 0, -1,
      4, 5, 6, 7, -1,
      0, 1, 5, 4, -1,
      1, 2, 6, 5, -1,
      2, 3, 7, 6, -1,
      3, 0, 4, 7, -1 ]
  }
\end{verbatim}

\subsection{Hints}
Comments can be used when hand crafting a VRML file; comment lines begin with a ``\verb~#~''. Due to
limitations of the KiCad VRML engine, do not put comments within any Transform node.

Use bevels; this may make an arduous task of describing even a simple component but there is a significant
change in the appearance of the final rendered product. Some experimentation will help you get a feel for
which corners can be left unbeveled without compromising the appearance.

When describing a large cylindrical object, use a polygon with at least 16 sides. The resulting component
may not look as smooth as a real cylinder but it will be pretty enough to use in the assembly model.

When describing a small cylindrical object such as a test pin, use an 8-sided polygon. The component may
look ugly but it will be good enough. If you think the component looks too ugly go ahead and use a 12 or
16-sided polygon.

When describing the wire leads of an object, use an 8-sided polygon and don't put bevels at the ends
of the wires. A 12-sided polygon will make a noticeable improvement to the wire but this may not be a
worthwhile improvement. Small wires may use 6-sided polygons with no significant deterioration in
appearance.

For many classes of components you can define one instance, parameterize the component, and write a simple
program to automatically generate models. The DIL packages are a good example of this; the length and
width of the package can be used as parameters to generate the case while the pin count and spacing can be
used to generate all the pins. To make the algorithm more general you might consider allowing different
thickness packages and parameterizing the pin so that it can change dimensions to suit the thickness. If you
had produced a beautiful component description which can be parameterized and used to generate a whole
family of useful 3D models, you may consider writing a program to do it.

\section{Example 3D Model: Electrolytic capacitor}
In this section we will create an electrolytic capacitor; the basic dimensions are a radius of 4mm, height of
7mm, radial lead spacing 3.5mm, lead diameter 0.45mm, trimmed lead length of 4mm. We will put a 0.5mm bevel on
either end of the cylinder. First we convert the dimensions into our virtual world units (0.1 inch = 1 unit):

\begin{tabular}{|r|l|l|}
\hline
Item & Size (mm) & Size (world) \\
\hline
radius & 4 & 1.574\\
height & 7 & 2.756\\
lead spacing & 3.5 & 1.378\\
lead dia. & 0.45 & 0.177\\
lead length & 4 & 1.574\\
bevel & 0.5 & 0.197\\
\hline
\end{tabular}

In this example we will use 3 different materials --- a blue cylinder for the capacitor, a black stripe to
indicate the negative side, and a silver material for the leads. We can use a spreadsheet to calculate the
vertices for each polygon describing the capacitor and its leads. Let us list the vertex points of each
polygon starting with the uppermost point as seen when we view the model from above (highest Y value) and
put the vertices in a counterclockwise order as seen from above.
We will use 16 sides for the capacitor's body and 6 sides for each lead; the radius for the beveled portion is
equal to the radius of the unbeveled portion minus the bevel size. The bottom of the capacitor will sit flush
with the board; a more realistic model would have a small gap between the board and the bottom of the
component. For the stripe we will need to add four points to the facet which will bear the stripe; when creating
the body facets we must not render the area of the stripe since a surface cannot have two appearances and this
will lead to rendering artifacts. Also keep in mind that the facets are flat so we need to perform a linear
interpolation to place the vertices for the stripe.  In this example I used 1/3 the width of a facet for the stripe
and I centered the stripe on the body facet. The facet with the stripe must be rendered as four trapezoidal facets
so that a gap is left for the stripe. In this model the negative lead is on the left.

\begin{tabular}{|l|l|l|l||l|l|l|l|}
\hline
\multicolumn{8}{|c|}{\textbf{Bevel and Body Vertices}}\\
\hline
Bevel X & Y & Lower Z & Upper Z & Body X & Y & Lower Z & Upper Z\\
\hline
 0.000 &  1.377 & 0    & 2.756 &  0.000 &  1.574 & 0.197 & 2.559\\
-0.527 &  1.272 & ---  & ---   & -0.602 &  1.454 & ---   & ---\\
-0.974 &  0.974 & ---  & ---   & -1.113 &  1.113 & ---   & ---\\
-1.272 &  0.527 & ---  & ---   & -1.454 &  0.602 & ---   & ---\\
-1.377 &  0.000 & ---  & ---   & -1.574 &  0.000 & ---   & ---\\
-1.272 & -0.527 & ---  & ---   & -1.454 & -0.602 & ---   & ---\\
-0.974 & -0.974 & ---  & ---   & -1.113 & -1.113 & ---   & ---\\
-0.527 & -1.272 & ---  & ---   & -0.602 & -1.454 & ---   & ---\\
 0.000 & -1.377 & ---  & ---   &  0.000 & -1.574 & ---   & ---\\
 0.527 & -1.272 & ---  & ---   &  0.602 & -1.454 & ---   & ---\\
 0.974 & -0.974 & ---  & ---   &  1.113 & -1.113 & ---   & ---\\
 1.272 & -0.527 & ---  & ---   &  1.454 & -0.602 & ---   & ---\\
 1.377 &  0.000 & ---  & ---   &  1.574 &  0.000 & ---   & ---\\
 1.272 &  0.527 & ---  & ---   &  1.454 &  0.602 & ---   & ---\\
 0.974 &  0.974 & ---  & ---   &  1.113 &  1.113 & ---   & ---\\
 0.527 &  1.272 & ---  & ---   &  0.602 &  1.454 & ---   & ---\\
\hline
\end{tabular}


\begin{tabular}{|l|l||l|l||l|l|}
\hline
\multicolumn{6}{|c|}{\textbf{Lead Vertices}}\\
\hline
Lead+ X & Y & Lead- X & Y & Lower Z & Upper Z\\
\hline
0.689 &  0.089 & -0.689 &  0.089 & 0   & -1.574\\
0.612 &  0.044 & -0.766 &  0.044 & --- & ---\\
0.612 & -0.044 & -0.766 & -0.044 & --- & ---\\
0.689 & -0.088 & -0.689 & -0.088 & --- & ---\\
0.766 & -0.044 & -0.612 & -0.044 & --- & ---\\
0.766 &  0.044 & -0.612 &  0.044 & --- & ---\\
\hline
\end{tabular}

Remember that when forming facets, the vertices must be enumerated so that they appear
in a counterclockwise order as viewed from the visible side.  We want to take one facet
near the negative lead and color it black for reference. The resulting VRML file may
look like the following; for a little exercise, try specifying the facets yourself.
Use a VRML viewer like whitedune frequently to ensure that you are placing the facet
indices in the correct order and that the indices are correct.

\begin{verbatim}
#VRML V2.0 utf8

# Capacitor Body
# Vertices are in the order:
#   lower bevel (0 .. 15)
#   lower body  (16 .. 31)
#   upper body  (32 .. 47)
#   upper bevel (48 .. 63)
#   stripe (64 .. 67)
# Facets are in the order:
#   lower polygon, upper polygon, lower bevel, upper bevel, body, stripe
DEF capbody Transform {
  children [
    Shape {
      appearance Appearance {
        material DEF cap_blue Material {
          diffuseColor 0.3 0.3 1
          emissiveColor 0 0 0
          specularColor 0.3 0.3 1
          ambientIntensity 0.7
          transparency 0
          shininess 0.3
        }
      }
      geometry IndexedFaceSet {
        coord Coordinate { point [
           0.000  1.377 0, -0.527  1.272 0, -0.974  0.974 0, -1.272  0.527 0,
          -1.377  0.000 0, -1.272 -0.527 0, -0.974 -0.974 0, -0.527 -1.272 0,
           0.000 -1.377 0,  0.527 -1.272 0,  0.974 -0.974 0,  1.272 -0.527 0,
           1.377  0.000 0,  1.272  0.527 0,  0.974  0.974 0,  0.527  1.272 0,
           0.000  1.574 0.197, -0.602  1.454 0.197,
          -1.113  1.113 0.197, -1.454  0.602 0.197,
          -1.574  0.000 0.197, -1.454 -0.602 0.197,
          -1.113 -1.113 0.197, -0.602 -1.454 0.197,
           0.000 -1.574 0.197,  0.602 -1.454 0.197,
           1.113 -1.113 0.197,  1.454 -0.602 0.197,
           1.574  0.000 0.197,  1.454  0.602 0.197,
           1.113  1.113 0.197,  0.602  1.454 0.197,
           0.000  1.574 2.559, -0.602  1.454 2.559,
          -1.113  1.113 2.559, -1.454  0.602 2.559,
          -1.574  0.000 2.559, -1.454 -0.602 2.559,
          -1.113 -1.113 2.559, -0.602 -1.454 2.559,
           0.000 -1.574 2.559,  0.602 -1.454 2.559,
           1.113 -1.113 2.559,  1.454 -0.602 2.559,
           1.574  0.000 2.559,  1.454  0.602 2.559,
           1.113  1.113 2.559,  0.602  1.454 2.559,
           0.000  1.377 2.756, -0.527  1.272 2.756,
          -0.974  0.974 2.756, -1.272  0.527 2.756,
          -1.377  0.000 2.756, -1.272 -0.527 2.756,
          -0.974 -0.974 2.756, -0.527 -1.272 2.756,
           0.000 -1.377 2.756,  0.527 -1.272 2.756,
           0.974 -0.974 2.756,  1.272 -0.527 2.756,
           1.377  0.000 2.756,  1.272  0.527 2.756,
           0.974  0.974 2.756,  0.527  1.272 2.756,
          -1.534 -0.200 0.697, -1.494 -0.400 0.697,
          -1.494 -0.400 2.059, -1.534 -0.200 2.059 ]
        }
        coordIndex [
          15,14,13,12,11,10,9,8,7,6,5,4,3,2,1,0,-1,
          48,49,50,51,52,53,54,55,56,57,58,59,60,61,62,63,-1,
          0,1,17,16,-1, 1,2,18,17,-1, 2,3,19,18,-1, 3,4,20,19,-1,
          4,5,21,20,-1, 5,6,22,21,-1, 6,7,23,22,-1, 7,8,24,23,-1,
          8,9,25,24,-1, 9,10,26,25,-1, 10,11,27,26,-1, 11,12,28,27,-1,
          12,13,29,28,-1, 13,14,30,29,-1, 14,15,31,30,-1, 15,0,16,31,-1,
          32,33,49,48,-1, 33,34,50,49,-1, 34,35,51,50,-1, 35,36,52,51,-1,
          36,37,53,52,-1, 37,38,54,53,-1, 38,39,55,54,-1, 39,40,56,55,-1,
          40,41,57,56,-1, 41,42,58,57,-1, 42,43,59,58,-1, 43,44,60,59,-1,
          44,45,61,60,-1, 45,46,62,61,-1, 46,47,63,62,-1, 47,32,48,63,-1,
          16,17,33,32,-1, 17,18,34,33,-1, 18,19,35,34,-1, 19,20,36,35,-1,
          21,22,38,37,-1, 22,23,39,38,-1, 23,24,40,39,-1,
          24,25,41,40,-1, 25,26,42,41,-1, 26,27,43,42,-1, 27,28,44,43,-1,
          28,29,45,44,-1, 29,30,46,45,-1, 30,31,47,46,-1, 31,16,32,47,-1,
          20,21,65,64,-1, 21,37,66,65,-1, 37,36,67,66,-1, 36,20,64,67,-1 ]
      }
    }
  ]
}

# Stripe on - side
DEF capbody Transform {
  children [
    Shape {
      appearance Appearance {
        material DEF cap_black Material {
          diffuseColor 0 0 0
          emissiveColor 0 0 0
          specularColor 0 0 0
          ambientIntensity 0.1
          transparency 0
          shininess 0.3
        }
      }
      geometry IndexedFaceSet {
        coord Coordinate { point [
          -1.534 -0.200 0.697, -1.494 -0.400 0.697,
          -1.494 -0.400 2.059, -1.534 -0.200 2.059 ]
        }
        coordIndex [
          0,1,2,3,-1 ]
      }
    }
  ]
}

# Positive lead
DEF capleadp Transform {
  children [
    Shape {
      appearance Appearance {
        material DEF cap_lead Material {
          diffuseColor 0.7 0.7 0.7
          emissiveColor 0 0 0
          specularColor 0.7 0.7 0.7
          ambientIntensity 0.8
          transparency 0
          shininess 0.6
        }
      }
      geometry IndexedFaceSet {
        coord Coordinate { point [
        0.689  0.089 0, 0.612  0.044 0, 0.612 -0.044 0,
        0.689 -0.088 0, 0.766 -0.044 0, 0.766  0.044 0,
        0.689  0.089 -1.574, 0.612  0.044 -1.574, 0.612 -0.044 -1.574,
        0.689 -0.088 -1.574, 0.766 -0.044 -1.574, 0.766  0.044 -1.574 ]
        }
        coordIndex [
          0,1,2,3,4,5,-1, 11,10,9,8,7,6,-1,
          1,0,6,7,-1, 2,1,7,8,-1, 3,2,8,9,-1,
          4,3,9,10,-1, 5,4,10,11,-1, 0,5,11,6,-1 ]
      }
    }
  ]
}

# Negative lead
DEF capleadm Transform {
  children [
    Shape {
      appearance Appearance {
        material USE cap_lead
      }
      geometry IndexedFaceSet {
        coord Coordinate { point [
          -0.689  0.089 0, -0.766  0.044 0, -0.766 -0.044 0,
          -0.689 -0.088 0, -0.612 -0.044 0, -0.612  0.044 0,
          -0.689  0.089 -1.574, -0.766  0.044 -1.574, -0.766 -0.044 -1.574,
          -0.689 -0.088 -1.574, -0.612 -0.044 -1.574, -0.612  0.044 -1.574 ]
        }
        coordIndex [
          0,1,2,3,4,5,-1, 11,10,9,8,7,6,-1,
          1,0,6,7,-1, 2,1,7,8,-1, 3,2,8,9,-1,
          4,3,9,10,-1, 5,4,10,11,-1, 0,5,11,6,-1 ]
      }
    }
  ]
}
\end{verbatim}


\section{A Parametric Description of the DIL Package}
In this section I will describe the process of creating a parametric model to generate
VRML files for use in KiCad.  I have chosen the DIL package since it provides a simple
example; every DIL package from a series is the same except for the length of the case
and the number of pins. The pins are identical except for a translation and rotation
operation, so to describe an entire family of DIL packages we only need to describe a
single pin and the case.

 In this example I imagine the pin as consisting of a horizontal, vertical, and a 45 degree
 portion. The pin is beveled along all edges except for where it attaches to the case.
 Figure~\ref{fig:dilpin} shows the vertices comprising a single pin, Table~\ref{tab:dilparams}
 lists the parameters needed to describe a generic DIL package and Table~\ref{tab:pinvertices}
 shows the equations for each vertex. Additional parameters needed to position the pins
 correctly are the pin spacing and the number of pins. If we have N pins, $\mathrm{Pin}_1$
 is located at $X = [(2-N)/4*\mathrm{spacing}]$ and $\mathrm{Pin}_{N/2+1}$ is rotated 180 degrees
 about the Z axis and located at $X = [(N-2)/4*\mathrm{spacing}]$.

\begin{table}
\caption{Parameters describing a DIL package.}
\label{tab:dilparams}
\begin{tabular}{|l|l|}
\hline
\textbf{Parameter} & \textbf{Description}\\
\hline
D & Case length\\
E1 & Case width\\
M & Depth of untapered middle section of case; pins attach here.\\
A1 & Case base height; distance between the PCB and bottom of the case.\\
A2 & Case depth\\
S & Case taper (DILs typically taper from the middle as seen from the side)\\
NW & Notch width\\
NL & Notch length\\
ND & Notch depth\\
CB & Case bevel\\
\hline
%\multicolumn{2}{|c|}{}\\
\hline
PB & Pin bevel\\
B & Pin width through PCB\\
B1 & Pin width at case\\
B2 & Length of tapered portion\\
C1 & Length of top bent portion (where the leg bends from horizontal to vertical)\\
C2 & Offset of lower bent portion\\
C  & Pin thickness\\
E  & Pin row spacing\\
L  & Pin length (narrow part which goes through PCB)\\
E2 & Pin-to-pin spacing\\
\hline
\end{tabular}
\end{table}

\begin{figure}
\begin{center}
\includegraphics[height=0.8\textheight]{eps/pin.eps}
\end{center}
\caption{Vertices describing a beveled DIL pin.}
\label{fig:dilpin}
\end{figure}

\begin{longtable}{|l|l|l|l|}
\caption{Equations for vertices of a generic DIL pin\label{tab:pinvertices}}\\
\hline
& \multicolumn{3}{|c|}{\raisebox{0pt}[2.5ex][5pt]{\textbf{Equation}}}\\\cline{2-4}
\textbf{Vertex} & \raisebox{0pt}[2.5ex][5pt]{\textbf{X}} & \textbf{Y} &
\textbf{Z}\endhead
\hline
&\multicolumn{3}{|c|}{\raisebox{0pt}[2.5ex][5pt]{\textbf{Upper Top}}}\\\cline{2-4}
\raisebox{0pt}[2.5ex][0pt]{0}
   & $-B1/2$   & $0$                 & $A1+(A2+C)/2$\\
1  & $-B1/2$   & $(E1-E-C)/2+C1$     & $A1+(A2+C)/2$\\
2  & $B1/2$    & $(E1-E-C)/2+C1$     & $A1+(A2+C)/2$\\
3  & $B1/2$    & $0$                 & $A1+(A2+C)/2$\\
\cline{2-4}
&\multicolumn{3}{|c|}{\raisebox{0pt}[2.5ex][5pt]{\textbf{Front}}}\\\cline{2-4}
\raisebox{0pt}[2.5ex][0pt]{4}
   & $-B1/2$   & $(E1-E-C)/2$        & $A1+(A2+C)/2-C1$\\
5  & $-B1/2$   & $(E1-E-C)/2$        & $B2$\\
6  & $-B/2$    & $(E1-E-C)/2$        & $0$\\
7  & $-B/2$    & $(E1-E-C)/2$        & $PB-L$\\
8  & $B/2$     & $(E1-E-C)/2$        & $PB-L$\\
9  & $B/2$     & $(E1-E-C)/2$        & $0$\\
10 & $B1/2$    & $(E1-E-C)/2$        & $B2$\\
11 & $B1/2$    & $(E1-E-C)/2$        & $A1+(A2+C)/2-C1$\\
\cline{2-4}
&\multicolumn{3}{|c|}{\raisebox{0pt}[2.5ex][5pt]{\textbf{Lower Top}}}\\\cline{2-4}
\raisebox{0pt}[2.5ex][0pt]{12}
   & $-B1/2$   & $0$                 & $A1+(A2-C)/2$\\
13 & $-B1/2$   & $(E1-E+C)/2+C2$     & $A1+(A2-C)/2$\\
14 & $B1/2$    & $(E1-E+C)/2+C2$     & $A1+(A2-C)/2$\\
15 & $B1/2$    & $0$                 & $A1+(A2-C)/2$\\
\cline{2-4}
&\multicolumn{3}{|c|}{\raisebox{0pt}[2.5ex][5pt]{\textbf{Rear}}}\\\cline{2-4}
\raisebox{0pt}[2.5ex][0pt]{16}
   & $-B1/2$   & $(E1-E+C)/2$        & $A1+(A2-C)/2-C2$\\
17 & $-B1/2$   & $(E1-E+C)/2$        & $B2$\\
18 & $-B/2$    & $(E1-E+C)/2$        & $0$\\
19 & $-B/2$    & $(E1-E+C)/2$        & $PB-L$\\
20 & $B/2$     & $(E1-E+C)/2$        & $PB-L$\\
21 & $B/2$     & $(E1-E+C)/2$        & $0$\\
22 & $B1/2$    & $(E1-E+C)/2$        & $B2$\\
23 & $B1/2$    & $(E1-E+C)/2$        & $A1+(A2-C)/2-C2$\\
\cline{2-4}
&\multicolumn{3}{|c|}{\raisebox{0pt}[2.5ex][5pt]{\textbf{Left Face}}}\\\cline{2-4}
\raisebox{0pt}[2.5ex][0pt]{24}
   & $-B1/2-PB$ & $0$                & $A1+(A2-C)/2+PB$\\
25 & $-B1/2-PB$ & $(E1-E+C)/2+C2-PB$ & $A1+(A2-C)/2+PB$\\
26 & $-B1/2-PB$ & $(E1-E+C)/2-PB$    & $A1+(A2-C)/2+PB-C2$\\
27 & $-B1/2-PB$ & $(E1-E+C)/2-PB$    & $B2$\\
28 & $-B1/2-PB$ & $(E1-E+C)/2-PB$    & $0$\\
29 & $-B1/2-PB$ & $(E1-E+C)/2-PB$    & $PB-L$\\
30 & $-B1/2-PB$ & $(E1-E-C)/2+PB$    & $PB-L$\\
31 & $-B1/2-PB$ & $(E1-E-C)/2+PB$    & $0$\\
32 & $-B1/2-PB$ & $(E1-E-C)/2+PB$    & $A1+(A2-C)/2-C2$\\
33 & $-B1/2-PB$ & $(E1-E-C)/2+PB$    & $A1+(A2+C)/2-C1-PB$\\
34 & $-B1/2-PB$ & $(E1-E-C)/2+C1+PB$ & $A1+(A2+C)/2-PB$\\
35 & $-B1/2-PB$ & $0$                & $A1+(A2+C)/2-PB$\\
\cline{2-4}
&\multicolumn{3}{|c|}{\raisebox{0pt}[2.5ex][5pt]{\textbf{Right Face}}}\\\cline{2-4}
\raisebox{0pt}[2.5ex][0pt]{36}
   & $B1/2+PB$  & $0$                & $A1+(A2+C)/2-PB$\\
37 & $B1/2+PB$  & $(E1-E-C)/2+C1+PB$ & $A1+(A2+C)/2-PB$\\
38 & $B1/2+PB$  & $(E1-E-C)/2+PB$    & $A1+(A2+C)/2-C1-PB$\\
39 & $B1/2+PB$  & $(E1-E-C)/2+PB$    & $B2$\\
40 & $B1/2+PB$  & $(E1-E-C)/2+PB$    & $0$\\
41 & $B1/2+PB$  & $(E1-E-C)/2+PB$    & $PB-L$\\
42 & $B1/2+PB$  & $(E1-E+C)/2-PB$    & $PB-L$\\
43 & $B1/2+PB$  & $(E1-E+C)/2-PB$    & $0$\\
44 & $B1/2+PB$  & $(E1-E+C)/2-PB$    & $B2$\\
45 & $B1/2+PB$  & $(E1-E+C)/2-PB$    & $A1+(A2-C)/2-C2$\\
46 & $B1/2+PB$  & $(E1-E+C)/2+PB$    & $A1+(A2-C)/2+PB$\\
47 & $B1/2+PB$  & $0$                & $A1+(A2-C)/2+PB$\\
\cline{2-4}
&\multicolumn{3}{|c|}{\raisebox{0pt}[2.5ex][5pt]{\textbf{Bottom}}}\\\cline{2-4}
\raisebox{0pt}[2.5ex][0pt]{48}
   & $B/2-PB$  & $(E1-E-C)/2+PB$     & $-L$\\
49 & $-B/2+PB$ & $(E1-E-C)/2+PB$     & $-L$\\
50 & $-B/2+PB$ & $(E1-E+C)/2-PB$     & $-L$\\
51 & $B/2-PB$  & $(E1-E+C)/2-PB$     & $-L$\\
\hline
\end{longtable}


\begin{figure}
\begin{center}
\includegraphics[height=0.8\textheight]{eps/case.eps}
\end{center}
\caption{Vertices describing a beveled DIL case.}
\label{fig:dilcase}
\end{figure}


\begin{longtable}{|l|l|l|l|}
\caption{Equations for vertices of a generic DIL case\label{tab:casevertices}}\\
\hline
& \multicolumn{3}{|c|}{\raisebox{0pt}[2.5ex][5pt]{\textbf{Equation}}}\\\cline{2-4}
\textbf{Vertex} & \raisebox{0pt}[2.5ex][5pt]{\textbf{X}} & \textbf{Y} &
\textbf{Z}\endhead
\hline
&\multicolumn{3}{|c|}{\raisebox{0pt}[2.5ex][5pt]{\textbf{Bottom}}}\\\cline{2-4}
\raisebox{0pt}[2.5ex][0pt]{0}
   & $-D/2+S+CB$ & $-E1/2+S+CB$        & $A1$\\
1  & $D/2-S-CB$  & $-E1/2+S+CB$        & $A1$\\
2  & $D/2-S-CB$  & $E1/2-S-CB$         & $A1$\\
3  & $-D/2+S+CB$ & $E1/2-S-CB$         & $A1$\\
\hline
&\multicolumn{3}{|c|}{\raisebox{0pt}[2.5ex][5pt]{\textbf{Lower Bevel}}}\\\cline{2-4}
\raisebox{0pt}[2.5ex][0pt]{4}
   & $-D/2+S$    & $-E1/2+S+CB$        & $A1+CB$\\
5  & $-D/2+S+CB$ & $-E1/2+S$           & $A1+CB$\\
6  & $D/2-S-CB$  & $-E1/2+S$           & $A1+CB$\\
7  & $D/2-S$     & $-E1/2+S+CB$        & $A1+CB$\\
8  & $D/2-S$     & $E1/2-S-CB$         & $A1+CB$\\
9  & $D/2-S-CB$  & $E1/2-S$            & $A1+CB$\\
10 & $-D/2+S+CB$ & $E1/2-S$            & $A1+CB$\\
11 & $-D/2+S$    & $E1/2-S-CB$         & $A1+CB$\\
\hline
&\multicolumn{3}{|c|}{\raisebox{0pt}[2.5ex][5pt]{\textbf{Lower Midsection}}}\\\cline{2-4}
\raisebox{0pt}[2.5ex][0pt]{12}
   & $-D/2$      & $-E1/2+CB$          & $A1+(A2-M)/2$\\
13 & $-D/2+CB$   & $-E1/2$             & $A1+(A2-M)/2$\\
14 & $D/2-CB$    & $-E1/2$             & $A1+(A2-M)/2$\\
15 & $D/2$       & $-E1/2+CB$          & $A1+(A2-M)/2$\\
16 & $D/2$       & $E1/2-CB$           & $A1+(A2-M)/2$\\
17 & $D/2-CB$    & $E1/2$              & $A1+(A2-M)/2$\\
18 & $-D/2+CB$   & $E1/2$              & $A1+(A2-M)/2$\\
19 & $-D/2$      & $E1/2-CB$           & $A1+(A2-M)/2$\\
\hline
&\multicolumn{3}{|c|}{\raisebox{0pt}[2.5ex][5pt]{\textbf{Upper Midsection}}}\\\cline{2-4}
\raisebox{0pt}[2.5ex][0pt]{20}
   & $-D/2$      & $-E1/2+CB$          & $A1+(A2+M)/2$\\
21 & $-D/2+CB$   & $-E1/2$             & $A1+(A2+M)/2$\\
22 & $D/2-CB$    & $-E1/2$             & $A1+(A2+M)/2$\\
23 & $D/2$       & $-E1/2+CB$          & $A1+(A2+M)/2$\\
24 & $D/2$       & $E1/2-CB$           & $A1+(A2+M)/2$\\
25 & $D/2-CB$    & $E1/2$              & $A1+(A2+M)/2$\\
26 & $-D/2+CB$   & $E1/2$              & $A1+(A2+M)/2$\\
27 & $-D/2$      & $E1/2-CB$           & $A1+(A2+M)/2$\\

\hline
\multicolumn{2}{|l|}{Height ($H$)}  & \multicolumn{2}{|l|}{$A1+A2-ND-2*CB$}\\
\multicolumn{2}{|l|}{Slope ($G$)}   & \multicolumn{2}{|l|}{$2*S/(A2-M-2*BEV)$}\\
\multicolumn{2}{|l|}{Offset ($O$)}  & \multicolumn{2}{|l|}{$(H-(A1+(A2+M)/2))*G$}\\
\hline
&\multicolumn{3}{|c|}{\raisebox{0pt}[2.5ex][5pt]{\textbf{Lower Notch Bevel}}}\\\cline{2-4}
\raisebox{0pt}[2.5ex][0pt]{28}
   & $-D/2+O$       & $NW/2+CB$     & $H$\\
29 & $-D/2+O$       & $-NW/2-CB$    & $H$\\
\hline
&\multicolumn{3}{|c|}{\raisebox{0pt}[2.5ex][5pt]{\textbf{Lower Notch}}}\\\cline{2-4}
\raisebox{0pt}[2.5ex][0pt]{30}
   & $-D/2+O+CB$                      & $NW/2$                & $H+CB$\\
31 & $-D/2+NL-NW/2$                   & $NW/2$                & $H+CB$\\
32 & $-D/2+NL-NW/2*(1-\cos(3/8*\pi))$ & $NW/2*\sin(3/8*\pi)$  & $H+CB$\\
33 & $-D/2+NL-NW/2*(1-\cos(\pi/4))$   & $NW/2*\sin(\pi/4)$    & $H+CB$\\
34 & $-D/2+NL-NW/2*(1-\cos(\pi/8))$   & $NW/2*\sin(\pi/8)$    & $H+CB$\\
35 & $-D/2+NL-NW/2*(1-\cos(\pi/8))$   & $-NW/2*\sin(\pi/8)$   & $H+CB$\\
36 & $-D/2+NL-NW/2*(1-\cos(\pi/4))$   & $-NW/2*\sin(\pi/4)$   & $H+CB$\\
37 & $-D/2+NL-NW/2*(1-\cos(3/8*\pi))$ & $-NW/2*\sin(3/8*\pi)$ & $H+CB$\\
38 & $-D/2+NL-NW/2$                   & $-NW/2$               & $H+CB$\\
39 & $-D/2+O+CB$                      & $-NW/2$               & $H+CB$\\
\hline
&\multicolumn{3}{|c|}{\raisebox{0pt}[2.5ex][5pt]{\textbf{Upper Notch Bevel}}}\\\cline{2-4}
\raisebox{0pt}[2.5ex][0pt]{40}
   & $-D/2 + S$                       & $NW/2 + CB$           & $A1+A2-CB$\\
41 & $-D/2 + S + CB$                  & $NW/2$                & $A1+A2-CB$\\
42 & $-D/2+NL-NW/2$                   & $NW/2$                & $A1+A2-CB$\\
43 & $-D/2+NL-NW/2*(1-\cos(3/8*\pi))$ & $NW/2*\sin(3/8*\pi)$  & $A1+A2-CB$\\
44 & $-D/2+NL-NW/2*(1-\cos(\pi/4))$   & $NW/2*\sin(\pi/4)$    & $A1+A2-CB$\\
45 & $-D/2+NL-NW/2*(1-\cos(\pi/8))$   & $NW/2*\sin(\pi/8)$    & $A1+A2-CB$\\
46 & $-D/2+NL-NW/2*(1-\cos(\pi/8))$   & $-NW/2*\sin(\pi/8)$   & $A1+A2-CB$\\
47 & $-D/2+NL-NW/2*(1-\cos(\pi/4))$   & $-NW/2*\sin(\pi/4)$   & $A1+A2-CB$\\
48 & $-D/2+NL-NW/2*(1-\cos(3/8*\pi))$ & $-NW/2*\sin(3/8*\pi)$ & $A1+A2-CB$\\
49 & $-D/2+NL-NW/2$                   & $-NW/2$               & $A1+A2-CB$\\
50 & $-D/2 + S + CB$                  & $-NW/2$               & $A1+A2-CB$\\
51 & $-D/2 + S$                       & $-NW/2 - CB$          & $A1+A2-CB$\\
\hline
&\multicolumn{3}{|c|}{\raisebox{0pt}[2.5ex][5pt]{\textbf{Upper Case Bevel}}}\\\cline{2-4}
\raisebox{0pt}[2.5ex][0pt]{52}
   & $-D/2+S$    & $-E1/2+S+CB$ & $A1+A2-CB$\\
53 & $-D/2+S+CB$ & $-E1/2+S$    & $A1+A2-CB$\\
54 & $D/2-S-CB$  & $-E1/2+S$    & $A1+A2-CB$\\
55 & $D/2-S$     & $-E1/2+S+CB$ & $A1+A2-CB$\\
56 & $D/2-S$     & $E1/2-S-CB$  & $A1+A2-CB$\\
57 & $D/2-S-CB$  & $E1/2-S$     & $A1+A2-CB$\\
58 & $-D/2+S+CB$ & $E1/2-S$     & $A1+A2-CB$\\
59 & $-D/2+S$    & $E1/2-S-CB$  & $A1+A2-CB$\\
\hline
&\multicolumn{3}{|c|}{\raisebox{0pt}[2.5ex][5pt]{\textbf{Upper Case Notch}}}\\\cline{2-4}
\raisebox{0pt}[2.5ex][0pt]{60}
   & $-D/2 + S + CB$                             & $(NW/2 + CB)$               & $A1+A2$\\
61 & $-D/2 + NL - NW/2$                          & $(NW/2 + CB)$               & $A1+A2$\\
62 & $-D/2 + NL - (NW/2 + CB)*(1-\cos(3/8*\pi))$ & $(NW/2 + CB)*\sin(3/8*PI)$  & $A1+A2$\\
63 & $-D/2 + NL - (NW/2 + CB)*(1-\cos(\pi/4))$   & $(NW/2 + CB)*\sin(PI/4)$    & $A1+A2$\\
64 & $-D/2 + NL - (NW/2 + CB)*(1-\cos(\pi/8))$   & $(NW/2 + CB)*\sin(PI/8)$    & $A1+A2$\\
65 & $-D/2 + NL - (NW/2 + CB)*(1-\cos(\pi/8))$   & $-(NW/2 + CB)*\sin(PI/8)$   & $A1+A2$\\
66 & $-D/2 + NL - (NW/2 + CB)*(1-\cos(\pi/4))$   & $-(NW/2 + CB)*\sin(PI/4)$   & $A1+A2$\\
67 & $-D/2 + NL - (NW/2 + CB)*(1-\cos(3/8*\pi))$ & $-(NW/2 + CB)*\sin(3/8*PI)$ & $A1+A2$\\
68 & $-D/2 + NL - NW/2$                          & $-(NW/2 + CB)$              & $A1+A2$\\
69 & $-D/2 + S + CB$                             & $-(NW/2 + CB)$              & $A1+A2$\\
\hline
&\multicolumn{3}{|c|}{\raisebox{0pt}[2.5ex][5pt]{\textbf{Upper Case}}}\\\cline{2-4}
\raisebox{0pt}[2.5ex][0pt]{70}
   & $-D/2 + S + CB$  & $-E1/2 + S + CB$ & $A1+A2$\\
71 & $-D/2 + NL + NW$ & $-E1/2 + S + CB$ & $A1+A2$\\
72 & $D/2 - S - CB$   & $-E1/2 + S + CB$ & $A1+A2$\\
73 & $D/2 - S - CB$   & $E1/2 - S - CB$  & $A1+A2$\\
74 & $-D/2 + NL + NW$ & $E1/2 - S - CB$  & $A1+A2$\\
75 & $-D/2 + S + CB$  & $E1/2 - S - CB$  & $A1+A2$\\
\hline
\end{longtable}

\end{document}
